\section{Introduction}
Technology in modern society is ubiquitous, economically explosive, and culturally male. The stereotype of the young white male software developer is still the norm in information technology (IT) fields, even after decades of intervention to increase diversity in tech. According to the National Center for Women in Tech, in 2015 women made up 57\% of the U.S. workforce but held just 25\% of professional computing jobs; they likewise received more than half of all U.S. bachelor's degrees but only 17\% of degrees in computer science, down from 37\% in 1985 \citep{NCWIT2016Women}. Despite the large body of research documenting the underrepresentation of women in IT (see \autoref{sec:lit} for a review), technology has not seen the same progress toward gender parity observed in the other STEM fields \citep{GlassEtAl2013Whats}.

Researchers have proposed many explanations for the persistent gender imbalance in IT\@. Some focus on early exposure to computing, some on the importance of formal programming education, and some on hiring practices within the tech industry. The data on computer science education and technical employment support all of these interpretations, but few studies discuss more than one at a time, and longitudinal data, following individuals through education to employment, is exceedingly uncommon.

In this paper, I present a project that uses interactive, web-based data visualization to combine these siloed sources of data and present a holistic picture of the state of women's representation within IT\@. Visualizations can present multiple, complex data sources side by side, which allows viewers to compare those data without needing expertise in data analysis. Presenting the visualizations in a web browser, using responsive design principles and standard web technologies, makes it available to a broad audience. It also provides an infinite canvas, so that the web resource can combine an easy-to-grasp overview with detailed exploration of each dataset.

The visualization website, available online at \url{https://thekatheri.net/tech-ladies/}, is designed to be accessible to viewers with any level of background knowledge, from women who work in the tech field to people with a casual interest who may never have considered diversity in tech culture. To achieve this, it combines simple, consistent visualizations of each dataset with approachable narrative discussions of research on gender and technology. It also provides links to resources for further information, including the original data sources, popular media coverage, and advocacy organizations.

I begin the paper by reviewing prior work in \autoref{sec:lit}, including the recent literature on gender in technology (\S\ref{sec:lit-gender}) and the literature on narrative data visualization (\S\ref{sec:lit-vis}). I then discuss the project's design goals in \autoref{sec:design}, followed by an outline of the iterative development and testing process in \autoref{sec:dev}. \autoref{sec:conclusion} concludes with a discussion and suggestions for future improvements.
