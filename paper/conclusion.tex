\section{Conclusion}\label{sec:conclusion}
The lack of gender diversity in technology is well established but poorly understood. Many research studies show that men significantly outnumber women at all levels of tech culture and are more confident in their own technical skills, despite a lack of evidence that men are any more skilled than women in these fields. These studies tend to address a single stage of education or career, however, and do not provide a sense of overall progress. In this paper, I described the construction of a resource to connect these disparate data sources on women's participation in technology. It provides an overview of the traditional tech education-to-career pipeline, along with narrative descriptions that provide commentary and context for that data. It enables viewers to answer basic questions about gender diversity in tech and provides resources for further exploration, from datasets to media coverage to corporate diversity policies.

Testing with members of the target audience showed that the website, in its current state, succeeds in its basic goals---to present disparate data simply and comparably, so that users with no background knowledge understand the basic problem of gender diversity in tech while users with more exposure to the topic can use the data to contextualize the ongoing discussion about gender. The website provides viewers with ample information to explore, presented clearly enough to facilitate that exploration. All final testers were able to correctly answer factual questions about the data and the narrative included in the webpage, and all found something of interest without prompting. Several testers requested the URL of the final resource to explore more at their leisure.

However, the iterative nature of user-centered design means that even while those user tests validates the overall design, they also highlighted many areas for refinement. Consequently, the project will continue to develop, with the inclusion of supplementary data, personal stories, and time-oriented displays to quickly assess whether gender diversity is improving.
