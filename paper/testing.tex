{\color{magenta}

\section{User Testing}\label{sec:testing}
Throughout the design and development of the project, I collected feedback and incorporated it into later iterations. Two people offered feedback on the early design sketches and on the wireframes; I recruited a total of eight informal testers for the actual website, four loosely corresponding to each persona, to browse the site and offer their impressions. Some of the resulting changes were based on their explicit feedback, while others were based on my observations as they browsed. This section presents the major changes introduced during this process.

\subsection{Page Structure}
In several early wireframes, I included annotations directing viewers to points of interest directly on the graph. However, the users who saw these wireframes said the annotations felt cluttered and were less certain, not more, how to interpret the underlying data. This led to the narrative boxes in between graphical displays, providing similar context without infringing on graph space. However, later testers noted that these narrative frames are text-heavy and may be visually uninteresting to people who engage primarily with the visualizations; they may be a good place to bring in smaller, supplementary data sets to explore.

I also rearranged the detail views so that the introductory text is above the graphs, rather than next to the year graph as in the wireframes. This alleviated problems reading the category graph when bars were too short: the increased vertical space allowed for even short bars to be clearly visible.

\subsection{Navigation}
The menu bar on the top of the page jumps between primary sections---the data views along the pipeline---without displaying the narrative sections in between. Because the information is all contained on a single page, users have always been able to scroll to view all of the content. However, the earliest versions of the webpage did not have an indicator besides the native scrollbar that this was an option. Consequently, two of the first testers navigated the page entirely by clicking on the menu bar and never saw the narrative frames.

After observing these tests, I added a purple downward-pointing arrow to the top panel (seen in the top screenshot of \autoref{fig:screenshots}). It clearly indicates scrollability; all later testers scrolled down the page without difficulty. When viewers reach the first narrative frame, the arrow disappears to avoid obscuring later content. This did not cause further problems with navigation, presumably because by this point viewers had realized that scrolling was the easiest way to navigate the page.

\subsection{Future Improvements}
Testing also revealed areas for improvement that have not yet been addressed. Several testers commented that they would like to see more variation between groups: because high school and employment are very different life stages, people were interested in the different ways the lack of gender parity affects high schoolers and working adults. A popular suggestion was to include stories of girls and women affected directly, while two testers asked about the kinds of interventions used at each stage. One commented that she would like to see more variety in the section layouts, with more comparison features in the overview and unique displays for each set of details.

Another common request was for more time-oriented displays. While each detail display includes a summary of the data by year, the trends for all groups are never presented in the same view. Three of the eight testers asked for a time-specific overview, while one suggested animating the existing overview to show year-by-year changes. All four of these testers expressed a strong interest in an annotated timeline, showing changes in demographics along with the dates of big tech companies' diversity policies, nonprofit events, and other related milestones.
}
